\documentclass{article}
\usepackage{iftex}
\ifluatex
	\usepackage{fontspec}
\else
	\usepackage[utf8]{inputenc}
\fi
\usepackage{amssymb}
\usepackage{amsmath}
\usepackage[ngerman]{babel}
\usepackage{csquotes}

\usepackage{xcolor} %Paket für Verwendung von Farben
\usepackage{graphicx} %Paket zur Einbindung von Grafiken
\usepackage{tikz} %Paket zur Erstellung von Vektorgrafiken und Diagrammen

\usepackage[amsmath, thmmarks, hyperref]{ntheorem} %Paket zur Definition mathematischer Umgebungen wie Satz, Definition, ...

\theoremstyle{plain}
\theoremheaderfont{\normalfont\bfseries}
\theorembodyfont{\itshape}
\theoremseparator{:}
\theorempreskip{\topsep}
\theorempostskip{\topsep}
\theoremindent0cm
\theoremnumbering{arabic}
%\theoremsymbol{\ensuremath{\square}}
\newtheorem*{Lemma}{Lemma}

\theoremstyle{plain}
\theoremheaderfont{\normalfont\bfseries}
\theorembodyfont{\upshape}
\theoremseparator{:}
\theorempreskip{\topsep}
\theorempostskip{\topsep}
\theoremindent0cm
\theoremnumbering{arabic}
\theoremsymbol{}
\newtheorem{Aufgabe}{Aufgabe}
\newtheorem*{Loesung}{Lösung}
\newtheorem*{Definition}{Definition}

\theoremstyle{plain}
\theoremheaderfont{\normalfont\bfseries}
\theorembodyfont{\upshape}
\theoremseparator{:}
\theorempreskip{\topsep}
\theorempostskip{\topsep}
\theoremindent0cm
\theoremnumbering{arabic}
\theoremsymbol{\ensuremath{\square}}
\newtheorem*{Proof}{Proof}

\usepackage[headheight=24pt,heightrounded]{geometry}
\usepackage{graphicx}
\usepackage{titlesec}
\usepackage{fancyhdr}
\usepackage{enumerate}

\usepackage{xcolor}
\definecolor{urlcolor}{RGB}{0,136,204}
\definecolor{linkcolor}{RGB}{204,0,34}
\usepackage[colorlinks,urlcolor=urlcolor,linkcolor=linkcolor]{hyperref}
\usepackage[colorinlistoftodos,prependcaption,textsize=tiny]{todonotes}
\usepackage{mathtools}

\usepackage{algorithm2e}
\usepackage{listings}
\usepackage{totcount}


\setlength{\parindent}{0pt}
\setlength{\parskip}{\baselineskip}

\newcommand{\sheet}[1]{
    \pagestyle{fancy}
    \fancyhead[L]{Lecture 2020 \\ Exercise Sheet #1}
    \fancyhead[R]{Max Mustermann (0071337) \\ Vorname Nachname (0070042)}
    
    \titleformat{\section}
    {\normalfont\Large\bfseries}{\stepcounter{totsec}Aufgabe #1.\thesection}{1em}{}
}
\newtotcounter{totsec}

% \renewcommand*\thesection{Aufgabe \arabic{section}}
% \renewcommand*\thesubsection{Teilaufgabe \alph{subsection}}

\newcommand{\firsttask}{1}
\newcommand{\numtasks}{\the\numexpr (0+\totvalue{totsec}) }


\makeatletter
\renewcommand{\Repeat}[1]{%
    \expandafter\@Repeat\expandafter{\the\numexpr #1\relax}%
}

\def\@Repeat#1{%
    \ifnum#1>0
        \expandafter\@@Repeat\expandafter{\the\numexpr #1-1\expandafter\relax\expandafter}%
    \else
        \expandafter\@gobble
    \fi
}
\def\@@Repeat#1#2{%
    \@Repeat{#1}{#2}#2%
}
\makeatother

\newcounter{headertask}
\setcounter{headertask}{\firsttask{}-1}
\newcommand{\insertpoints}{\begin{tikzpicture}[remember picture,overlay]
	\node[yshift=-11\baselineskip, xshift=0.7\paperwidth] at (current page.north west) [text width=5cm,text centered,above right]{\Large{\begin{tabular}{\Repeat{\numtasks}{c|}c}
		\Repeat{\numtasks}{\stepcounter{headertask}\the\value{headertask} & }
		$\displaystyle\sum$ \\ \hline \Repeat{\numtasks}{&} \end{tabular}}};
\end{tikzpicture}%	
}
